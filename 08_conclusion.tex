\chapter{おわりに}
本章では,本研究におけるまとめ及び今後の課題について述べる.
\section{まとめ}
本研究では,DDB のようなメモリ使用量が大きい VM の移送が長期化する問題に対し,
DDB と連動した VM ライブ移送を提案した.
提案手法では,VM のメモリサイズがライブ移送の性能に大きく影響を与えることに着目し,
DDB が保持するキャッシュを解放してそれらを転送対象から除外して
メモリ転送量を削減することで,移送時間の短縮を実現する.

本提案機構では,まず移送開始前にデータベースとしてのデータの整合性を保証しつつ,
最大限のメモリ削減効果を得るようにキャッシュの解放を行い,QEMU へのアドレス通知を行う,
また,MongoDB と QEMU 間のセマンティクスギャップを解消するためのアドレス変換と,
それによって構築されたビットマップによる移送中メモリ転送制御を行う.
そして移送完了後に,移送先ホストにおいて MongoDB 上のキャッシュの透過的な復元を行う.

提案手法を QEMU 10.2.0,MongoDB 8.2.2 に実装し,
従来の Pre-copy 手法と比較を行った結果,
最大で総移送時間を 56\%,メモリ転送量を 67\% 削減することに成功した.

\section{今後の課題}

\subsection{移送中の並行的なキャッシュ復元}
本提案手法における課題として,キャッシュ再構築の効率化が挙げられる.
現在の提案手法では,転送をスキップした MongoDB 上のキャッシュページは,
移送完了後に MongoDB が該当データにアクセスした際の,
WiredTiger のキャッシュミス処理として共有ストレージから再取得される.
この方法では,移送完了直後にキャッシュミスが多発し,
一時的に VM 上の MongoDB のパフォーマンスが低下してしまう.
そのため,移送処理中にネットワークを介したメモリ転送と並行して,
共有ストレージからのキャッシュデータのプリフェッチを行うことで,
移送完了後のパフォーマンス低下を最小限にする必要がある.

\subsection{高負荷環境下への適用}
本提案手法における課題として,書き込み負荷が高い状況への対応が挙げられる.
現在の提案手法は,静的な環境を前提とした実験を行っており,
書き込み負荷が高い状況に対する検証は行っていない.
MongoDB に対して書き込み負荷が高い状況では,
転送スキップ対象となったページに対して移送中に書き込みが発生し,
再転送が必要になるページが増加することが考えられる.
このとき,転送量の削減効果が減少するだけでなく,
キャッシュ解放処理のオーバヘッドが顕在化するリスクがある.
そのため本提案手法には,移送中に書き込まれたキャッシュページへの対応などを含む,
書き込み頻度が高い状況における処理などを追加する必要がある.

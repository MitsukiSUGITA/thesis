\par
仮想マシンとは物理マシンをソフトウェアによってエミュレートし,仮想化されたリソースを用いてオペレーティングシステム (OS) と
アプリケーションを実行する仮想的なコンピュータのことである.この技術によって複数のサーバを統合し,サーバの管理コストを削減することができ,
実際にGoogle Cloud Platformや,Amazon Web ServicesなどのReal-world なクラウド環境で使用されている.
そしてこの仮想マシンを実行中にその影響を最小限に抑えながら別マシンに移送する技術として仮想マシンライブ移送というものがある.
これは先にメモリを転送してから移送先に実行を移す Pre-copy 手法と移送先に実行を移してからメモリを転送する Post-copy 手法が
主に採用されており,負荷分散や,耐障害性,省電力やパフォーマンス強化のために利用されている.
しかし,メモリ使用量が大きいドキュメント DB の移送に時間がかかるという欠点がある.
使用するメモリ量が大きいと一度の反復処理に時間がかかり,その間に書き換えられたデータは再度転送しなければならない.
仮想マシンの性能の劣化を避けるように移送しても,移送が長期化すると移送完了時には VM の最適配置が変化してしまい,移送の効果が薄れてしまう.
\par
本研究では,ドキュメント DB と連動した仮想マシンの移送手法を提案する.
具体的には,メモリの転送をスキップして移送先のメモリをストレージから再構成することで ,仮想マシンの移送時間を短縮する.
提案する手法の移送の手順としてはまず,ストレージの内容は移送元と移送先の双方からアクセス可能という前提の下,
移送元の仮想マシンのメモリの転送を行わずクリアする.この時,クリアするページのメタデータをストレージに保存し,
移送先の仮想マシンの構築時にその情報をもとに移送元の仮想マシンと同じ状態に復元する.
この手法を,MongoDB のストレージエンジンである Wiredtiger に対して適用する.
\par
具体的には,移送元の仮想マシンにおいてキャッシュ上の B+tree内の全てのリーフページを退避キューに入れ,
強制的にページを退避させる.このときに,退避するページのデータベース名やキー情報をストレージに保存する.
そして移送先の仮想マシンでキャッシュ上の B+tree を再構成する際に,
共有ストレージに保存してあるメタデータをもとにページをフェッチすることで,移送元の仮想マシンと同一状態のB+tree を再構成する.
\par
現時点では,データベースのデータ量が増加するにつれてメモリ量も増加することを計測し,
キャッシュ上にあるページをすべてクリアするコードを作成した.今後は,クリアしたリーフページのメタデータから,
B+tree の再構成を行うコードを作成する.また,作成したコードを用いて仮想マシンのライブ移送を実行し,提案手法の性能を評価する.

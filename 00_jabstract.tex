%abstract in Japanese
仮想化技術とは物理マシンをソフトウェアによってエミュレートし,仮想化されたリソースを用いて
自身の OS やアプリケーションを物理マシンから独立した環境で
仮想マシン (VM) として動作させる技術であり,主にコンソリデーションに用いられる.
VM ライブ移送とは,実行中の VM を実行の影響を最小限に抑えながら別マシンに移送する技術であり,
負荷分散や,耐障害性,省電力やパフォーマンス強化のために利用される.
ライブ移送では,VM のメモリ内容や CPU の状態を移送先に転送する.
MongoDB のようなドキュメントデータベース (DDB) を実行している VM の移送に時間が要することが問題として存在する.
MongoDB のようなページへの書き込み頻度が高いアプリケーションは,
ダーティレートがメモリ転送速度を上回ると反復転送が収束せず
ダウンタイム中に転送する必要があるページ量も増加するため,
総移送時間とダウンタイムの両方が大幅に増加してしまう可能性がある.
本論文では,DDB と連動した仮想マシンの移送手法を提案する.
DDB 上のキャッシュをクリアしてそれらの転送をスキップすることで,VM の移送時間を短縮する.
本論文では,提案手法を QEMU 10.2.0,MongoDB 8.2.2 に実装した.
実験の結果,最大で総移送時間を 69\%,ダウンタイムを 53\% 削減することに成功した.
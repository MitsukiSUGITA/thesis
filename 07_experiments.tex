\chapter{実験}
本章では,提案手法による VM ライブ移送パフォーマンスの実験について述べる.

\section{実験環境}
実験環境は移送を行うホストマシン 2 台,共有ストレージ用のマシン 1 台から構成される.
実験環境の概要図を図\ref{fig:exp_machine}に,各マシンの構成を表\ref{tab:machine_spec}に示す.
本実験は,同一ハードウェア構成を持つ 2 台の移送用マシンと,
VM のディスクイメージを共有するための NFS サーバマシンで構成される.
移送用マシンの CPU は Intel Xeon E-2124 (3.30 GHz, 4 コア 4 スレッド) を,
メモリは 64GB を搭載している.
共有ストレージ用マシンの CPU はIntel Xeon Gold 5215 (2.50 GHz, 10 コア 20 スレッド) を,
メモリは 128GB を搭載している.
全てのマシンは 1 Gbps のイーサネットで接続されており,
OS として Ubuntu 24.04 (Linux 6.8.0) が稼働している.
VMM としては QEMU/KVM 10.2.0 をベースに提案機構を実装したものを使用する.
ライブ移送の対象となる VM には 2 vCPU を割り当て,
ゲスト OS はホストマシン同様,Ubuntu 24.04 (Linux 6.8.0) が稼働している.
また,VM のメモリサイズは本実験におけるパラメータとして変化させる.
そして,VM 上では DDB として提案機構を組み込んだ MongoDB 8.2.2 を動作させる.

\begin{figure}[H]
  \centering
  \includegraphics[width=0.8\linewidth]{figure/exp_machine.png}
  \caption{実験環境の概要図}
  \label{fig:exp_machine}
\end{figure}

\begin{table}[H] % [H]でその場所に固定、[htbp]なら自動配置
  \centering
  \caption{実験に使用した各マシンの構成} % 表のタイトルは「上」に書く
  \label{tab:machine_spec} % 参照用のラベル
  \begin{tabular}{ccc} \toprule % lは「左寄せ」cは「中央」の意味
    マシン & CPU & メモリ \\ \midrule\midrule
    ホストマシン     & Intel Xeon E-2124 CPU @ 3.30GHz (4C/4T) & 64GB \\
    ストレージマシン & Intel Xeon Gold 5215 CPU @ 2.50GHz (10C/20T) & 128GB \\
    仮想マシン      & 2vCPU & 可変 \\ \bottomrule
  \end{tabular}
\end{table}

\section{実験方法}
提案手法によるライブ移送を従来の Pre-copy 手法と比較し,性能を評価する.
実験の初期状態として,移送開始前に VM 上で動作する MongoDB の
WiredTiger キャッシュサイズを VM メモリ割り当て量の 95\% に設定する.
そして,キャッシュ内のダーティページ占有率が 95\% となるように書き込みと更新を行うことで,
VM 上で動作する MongoDB の大量データがメモリ上に保持されている状況を生成する.
この書き込み処理終了後のアイドル状態の VM に対して移送を実行し,
その際の総移送時間,メモリ転送量,転送ページ数,CPU 時間,ダウンタイムを計測する.
以上の移送実験において,両手法において VM に 8GB, 16GB, 32GB, 48GB, 60GB と,
割り当てるメモリサイズを変化させて実験を行う.
また,各条件での実験を計 5 回行い,それらの平均の実験結果として採用する.

\section{実験結果}
\subsection{総移送時間}
実験における総移送時間の結果を図\ref{fig:exp_migration_time}に示す.
実験結果より,全てのメモリサイズにおいて,
総移送時間は従来の Pre-copy 手法と比較して大幅に短縮され,約半分の時間で移送を完了した.
この総移送時間には,提案手法における MongoDB 上のキャッシュを解放する時間も含まれているため,
キャッシュ解放によるオーバヘッドを加味しても,
移送プロセス全体の大幅な高速化が実現できているといえる.
また,メモリサイズが大きくなるにつれて時間短縮の効果が顕著になり,
メモリサイズが 60GB の条件下では最大約 56\% の短縮を達成した.
このことから,特にメモリサイズが大きい環境において,提案手法の優位性が高まると考えられる.

\begin{figure}[H]
  \centering
  \includegraphics[width=0.8\linewidth]{figure/exp_migration_time.png}
  \caption{総移送時間}
  \label{fig:exp_migration_time}
\end{figure}


\subsection{メモリ転送量}
実験におけるメモリ転送量の結果を図\ref{fig:exp_transffer_data}に,
転送ページ数の結果を図\ref{fig:exp_transffer_page}に示す.
実験結果より,全てのメモリサイズにおいて提案手法はメモリ転送量,
転送ページ数ともに Pre-copy 手法と比較して大幅に削減できていることが確認された.
また,メモリ転送量と転送ページ数の削減率が概ね一致していることから,
提案手法による解放したキャッシュのページ転送スキップが,
転送の削減に寄与しているといえる.
そして,メモリサイズが大きくなるにつれて転送削減効果が増し,
メモリサイズが 60GB の条件下では最大約 67\% の削減を達成した.
このことから,MongoDB のキャッシュに割り当てられるメモリサイズが大きくなるほど,
提案手法の効果が発揮されると考えられる.

\begin{figure}[H]
  \centering
  \includegraphics[width=0.8\linewidth]{figure/exp_transffer_data.png}
  \caption{メモリ転送量}
  \label{fig:exp_transffer_data}
\end{figure}

\begin{figure}[H]
  \centering
  \includegraphics[width=0.8\linewidth]{figure/exp_transffer_page.png}
  \caption{転送ページ数}
  \label{fig:exp_transffer_page}
\end{figure}


\subsection{CPU 時間}
実験におけるCPU 時間の結果を図\ref{fig:exp_cpu_time}に示す.
実験結果によると,全てのメモリサイズにおいて提案手法は Pre-copy 手法と比較して,
CPU 時間を一貫して削減できていることが確認された.
具体的には,どのメモリサイズにおいても 概ね 15\% 前後の削減率を示しており,
メモリサイズが 16GB の条件においては,最大約 21\% の削減を達成した.
提案手法では,MongoDB 側でのキャッシュ解放と,QEMU 側での転送スキップビットマップ構築
という追加の CPU 処理が発生する.
しかし,これらのオーバヘッドを加味してもページ転送処理の削減効果が上回るため,
結果的にトータルの CPU 時間が削減されていると考えられる.
このことから,メモリサイズに関わらず安定して従来手法より CPU 負荷が低く抑えられており,
メモリサイズが増加しても提案手法の CPU に対する効率性が損なわれないと考えられる.

\begin{figure}[H]
  \centering
  \includegraphics[width=0.8\linewidth]{figure/exp_cpu_time.png}
  \caption{CPU 時間}
  \label{fig:exp_cpu_time}
\end{figure}


\subsection{ダウンタイム}
実験におけるダウンタイムの結果を図\ref{fig:exp_downtime}に示す.
実験結果によると,全てのメモリサイズにおいて提案手法は Pre-copy 手法と比較して,
ダウンタイムを一貫して削減できていることが確認された.
特に,メモリサイズが 48GB の条件においては,最大約 42\% の削減を達成した.
これは,解放されたキャッシュの転送スキップにより,
Stop-and-Copy フェーズで転送すべきページ数が抑制されたためだと考えられる.
このことから,提案手法はどのメモリサイズにおいてもダウンタイムを抑制し,
サービス停止による影響を最小限に抑えることができると考えられる.

\begin{figure}[H]
  \centering
  \includegraphics[width=0.8\linewidth]{figure/exp_downtime.png}
  \caption{ダウンタイム}
  \label{fig:exp_downtime}
\end{figure}


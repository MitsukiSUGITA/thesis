
\newif\ifuplatex
%\uplatextrue
\ifuplatex
\documentclass[bachelor,dvipdfmx]{style/uplatex_thesis} % 卒業論文
%\documentclass[master,dvipdfmx]{style/uplatex_thesis} % 修士論文
%\documentclass[doctor,dvipdfmx]{style/uplatex_thesis} % 博士論文
\else
\documentclass[bachelor]{style/thesis} % 卒業論文
%\documentclass[master,hyperref]{style/thesis} % 修士論文
%\documentclass[doctor]{style/thesis} % 博士論文
\fi
% フォントのエンコード
\usepackage[T2A,T1]{fontenc}

% 数式のシンボルフォント。アメリカ数学会の設定
\usepackage{amsmath,amssymb}

% 色
\usepackage{color}
\usepackage{bm}

% 数詞
\usepackage[super]{nth}

% 囲み枠
\usepackage{ascmac}
\usepackage{fancybox}

% float 環境関連(図表)
\usepackage{float}
\usepackage{longtable,booktabs}
\usepackage{tabularx}

% URL の記述
\usepackage{url}

% OTF フォント(和文フォント)
\ifuplatex
\else
\usepackage{luacode}
\usepackage{luatexja-otf}
\usepackage[haranoaji,deluxe,jis2004]{luatexja-preset}
\fi
% \usepackage[ipa,jis2004]{luatexja-preset} % IPA フォント
%\usepackage[ipaex,jis2004]{luatexja-preset} % IPAex フォント
%\usepackage[sourcehan,deluxe,jis2004]{luatexja-preset} % 源ノ角/源ノ明朝フォント
%\usepackage[hiragino-pron,deluxe,jis2004]{luatexja-preset} % macOS ならこっちでも可(ヒラギノフォント)

% フォント設定(数式)
\ifuplatex
\else
\usepackage[math-style=ISO,bold-style=ISO]{unicode-math}
\fi
% フォント設定(欧文)
% main ... 本文フォント。ローマン体。\textrm で使用
% sans ... ヒゲのないフォント。サンセリフ体(和文で言うゴシック体)。\textsf で使用
% mono ... 固定幅フォント。タイプライタ体。ソースコードのリストに使う。\texttt で使用
% math ... 数式フォント
%
%\setmathfont{TeX Gyre Termes Math}
%\setmathfont{TeX Gyre Pagella Math}
%\setmainfont[Ligatures=TeX, Scale=0.95]{TeX Gyre Termes}
%\setmainfont[Ligatures=TeX, Scale=0.95]{TeX Gyre Pagella}
%\setsansfont[Ligatures=TeX, Scale=0.95]{TeX Gyre Heros}
%\setsansfont[Ligatures=TeX, Scale=0.9]{TeX Gyre Adventor}
%\setmonofont[Ligatures=TeX, Scale=1]{TeX Gyre Cursor}

% TeX Gyre フォントと OTF フォントの共存
%\renewcommand{\bfdefault}{bx}
%\renewcommand{\headfont}{\gtfamily\sffamily\bfseries}

\ifuplatex
\else
\usepackage{upquote}
% \usepackage[utf8x]{luainputenc} % 入力の文字エンコード
\fi

% 欧文組版のマイクロタイポグラフィー(細かな読み易さの調整)を有効化
\usepackage{microtype}

% 参考文献に BibLaTeX を使う
\usepackage[backend=bibtex, style=numeric, sorting=none]{biblatex}
\addbibresource{citation/reference.bib}

% PDF にハイパーリンク作成
\ifuplatex
\usepackage[dvipdfmx]{hyperref}
\else
\usepackage{hyperref}
\fi
% PDF のメタデータ設定
\hypersetup{%
            pdfauthor={杉田 光希},
            pdftitle={ドキュメント DB と連動する仮想マシンライブ移送},
            pdfsubject={2025年度 卒業論文},
            %pdfkeywords={クラウド; ライブラリ OS; 仮想化; 計測; Cloud; Library OS; virtualization; measurement; Unikernel},
}

%
% 題目
%
\title{ドキュメント DB と連動する仮想マシンライブ移送}
\etitle{Document DB-assisted Live Migration of Virtual Machines}
\absttitlestr{ドキュメント DB と連動する仮想マシンライブ移送} % 概要のページ用
\eabsttitlestr{Document DB-assisted Live Migration of Virtual Machines} % 概要のページ用

%
% 氏名
%
\jauthor{杉田 光希}
\eauthor{SUGITA Mitsuki}
% 学籍番号
\studentnumber{22266062}
% 所属
\institute{知能情報システム工学科}
%
% 指導教官
%
\supervisor{山田 浩史 准教授}
%
% 論文提出日
%
\deadline{2026年1月30日}



\pagestyle{plain}
% 文書開始
\begin{document}
% 日本語の概要はここに記述
\begin{jabstract}
\par
仮想マシンとは物理マシンをソフトウェアによってエミュレートし,仮想化されたリソースを用いてオペレーティングシステム (OS) と
アプリケーションを実行する仮想的なコンピュータのことである.この技術によって複数のサーバを統合し,サーバの管理コストを削減することができ,
実際にGoogle Cloud Platformや,Amazon Web ServicesなどのReal-world なクラウド環境で使用されている.
そしてこの仮想マシンを実行中にその影響を最小限に抑えながら別マシンに移送する技術として仮想マシンライブ移送というものがある.
これは先にメモリを転送してから移送先に実行を移す Pre-copy 手法と移送先に実行を移してからメモリを転送する Post-copy 手法が
主に採用されており,負荷分散や,耐障害性,省電力やパフォーマンス強化のために利用されている.
しかし,メモリ使用量が大きいドキュメント DB の移送に時間がかかるという欠点がある.
使用するメモリ量が大きいと一度の反復処理に時間がかかり,その間に書き換えられたデータは再度転送しなければならない.
仮想マシンの性能の劣化を避けるように移送しても,移送が長期化すると移送完了時には VM の最適配置が変化してしまい,移送の効果が薄れてしまう.
\par
本研究では,ドキュメント DB と連動した仮想マシンの移送手法を提案する.
具体的には,メモリの転送をスキップして移送先のメモリをストレージから再構成することで ,仮想マシンの移送時間を短縮する.
提案する手法の移送の手順としてはまず,ストレージの内容は移送元と移送先の双方からアクセス可能という前提の下,
移送元の仮想マシンのメモリの転送を行わずクリアする.この時,クリアするページのメタデータをストレージに保存し,
移送先の仮想マシンの構築時にその情報をもとに移送元の仮想マシンと同じ状態に復元する.
この手法を,MongoDB のストレージエンジンである Wiredtiger に対して適用する.
\par
具体的には,移送元の仮想マシンにおいてキャッシュ上の B+tree内の全てのリーフページを退避キューに入れ,
強制的にページを退避させる.このときに,退避するページのデータベース名やキー情報をストレージに保存する.
そして移送先の仮想マシンでキャッシュ上の B+tree を再構成する際に,
共有ストレージに保存してあるメタデータをもとにページをフェッチすることで,移送元の仮想マシンと同一状態のB+tree を再構成する.
\par
現時点では,データベースのデータ量が増加するにつれてメモリ量も増加することを計測し,
キャッシュ上にあるページをすべてクリアするコードを作成した.今後は,クリアしたリーフページのメタデータから,
B+tree の再構成を行うコードを作成する.また,作成したコードを用いて仮想マシンのライブ移送を実行し,提案手法の性能を評価する.

\end{jabstract}

% 英語の概要はここに記述
\begin{eabstract}
%abstract in English
Live migration of virtual machines (VMs) is a technique that moves active VMs between different physical hosts without losing any running states. 
Live migration is helpful to intra-datacenter administration including load balancing, power saving, and effective software upgrades.
Although it is desirable for administrators that the live migration is completed as quickly as possible, the pre-copy-based live migration, 
widely used in modern hypervisors, does not satisfy this demand on the current trend that VMs on which running applications are performance-critical 
such as document databases (DDBs) have quite large memory. 
This thesis presents an approach shortens the time for live-migrating VMs with even large memory DDBs. 
To quickly produce the running state of the migrating VMs on the destination, our approach performs regular memory transfers 
while simultaneously constructing the DDB's buffer-pool by fetching the data items from the shared storage. 
By conducting both transfers, the approach efficiently produces the running state of the target VM on the destination hosts.
We implemented a prototype on MongoDB 8.2.2, QEMU 10.2.0, and Linux 6.8.0. 
The experimental results show that the migration time of the prototype is up to 2.25{\texttimes} compared to the conventional pre-copy.

\end{eabstract}

\maketitle

% 本文
\chapter{はじめに}

仮想化技術とは物理マシンをソフトウェアによってエミュレートし,仮想化されたリソースを用いて
自身の OS やアプリケーションを物理マシンから独立した環境で仮想マシン (VM) として動作させる
技術である.仮想マシンモニタ (VMM) というソフトウェアによって VM の作成,管理が行われる.
VM は単一の物理マシン上で複数構築することが可能であるため主にコンソリデーションに用いられ,
複数のサーバを統合することでサーバの管理コストを削減することができる.
実際に,Google Cloud Platform \cite{CloudComputingServices}や Amazon Web Services \cite{CloudComputingServicesa},
Microsoft Azure \cite{CloudComputingServicesb}など Real-world なクラウド環境で使用されている.

仮想マシンライブ移送は,VM を稼働しながらその実行の影響を最小限に抑えつつ
別ホストに移送する技術である.
サーバー統合,ゼロダウンタイムでのハードウェアメンテナンス,エネルギー管理,トラフィック管理など,
クラウド管理業務の大部分は VM ライブ移送によって支えられている\cite{zhangSurveyVirtualMachine2018}.
例えば,Google のデータセンタ\cite{HomepageNdashGoogle}においても,
大規模なクラスタ管理やメンテナンスのためにライブ移送技術が日常的に利用されている.
移送手法には,先に移送先ホストにメモリを転送してから VM の実行権を移す Pre-copy 手法\cite{clarkLiveMigrationVirtuala}と,
先に VM の実行権を移送先ホストに移してからメモリを転送する Post-copy 手法\cite{hinesPostcopyLiveMigration2009}がある.
Pre-copy 手法は耐障害性が優れているが,反復的なメモリ転送によって総移送時間が長期化する傾向がある.
Post-copy 手法は短時間での移送が可能だが,移送中に移送先ホストで障害が発生すると VM が
壊れてしまうという致命的なリスクも伴う.
このように,両手法には信頼性と即時性でトレードオフが存在するため,
移送する環境によって移送手法を変えることが望ましい.

ドキュメントデータベース (DDB) はキーと値のペアをドキュメントで管理する,
NoSQL データベースの一種である.
DDB は, JSON 形式を用いた柔軟なスキーマ設計により複雑なデータを効率的に管理でき,
非正規化と分散設計によって低レイテンシかつ高いスケーラビリティを実現する\cite{akoushPredictingPerformanceVirtual2010}.
そのため,ビッグデータ処理やクラウド環境で重要な役割を担う\cite{carvalhoPerformanceEvaluationNoSQL2023}.
リアルタイム分析,ロギング,そしてブログのような小規模で柔軟な Web サイトで使用されており\cite{hechtNoSQLEvaluationUse2011},
代表的な DDB に MongoDB \cite{MongoDB} や CouchDB \cite{ApacheCouchDB}がある.

VM ライブ移送における問題に,MongoDB のような DDB などのメモリ集約型アプリケーションを
実行している VM の移送に時間が要することが挙げられる.
メモリサイズが大きい VM の移送は反復転送が長期化することで総移送時間も増加する傾向がある.
さらに,MongoDB のようなページへの書き込み頻度が高いアプリケーションは,
ダーティレートがメモリ転送速度を上回ると反復転送が収束せずダウンタイム中に転送する必要があるページ量も増加するため,
総移送時間とダウンタイムの両方が大幅に増加してしまう可能性がある.
実際にAmazon Document DB \cite{AmazonDocumentDBServerless} において,
最大 768GB ものメモリを使用可能であり,DDB を稼働させる VM のメモリサイズは肥大化している.
移送が長期化すると,移送完了時にはホストの稼働状況が変化してしまい,
想定していた負荷分散の効果が得られなくなるリスクがある.
そのためこのような大規模なメモリを持つ VM の移送遅延は解決すべき重要な課題となっている.

本論文では,DDB と連動した仮想マシンの移送手法を提案する.
提案手法では,VM のメモリサイズがライブ移送の性能に影響を与える主要因の一つであることと,
DDB は高スループット維持のために大量のデータをキャッシュとして保持する特性に着目し,
DDB が保持するキャッシュを解放してそれらの転送をスキップしてメモリ転送を削減することで,
移送時間の短縮を図る.
提案手法を実現するために,DDB と VMM を連携する機構を実装する.
具体的には,移送開始時に DDB で自身の保持するキャッシュを解放し,
その情報を移送を管理する VMM に通知して,
解放したキャッシュページの転送をスキップすることで転送量を削減する.
これにより DDB のメモリ使用量が大きく,ダーティレートが高い状況下でも
転送データ量を少量に保つことができ,総移送時間とダウンタイム双方の短縮を実現する.
DDB が稼働している VM においても短時間の移送を実現するため,
ライブ移送による負荷分散や電力削減といった効果を得る.

本論文の貢献は以下の通りである.
\begin{itemize}
  \item DDB が稼働している VM のライブ移送時間を短縮する手法を提案した.
  DDB が保持するキャッシュを解放して転送対象から除外し,
  共有ストレージから再構築することで,メモリ転送量を削減する.
  DDB のセマンティクスを利用して再構築可能なキャッシュ領域の転送をスキップするため,
  DDB のダーティレートが高い状況やメモリ使用量が大きい状況でも,
  既存手法と比べてキャッシュサイズに応じた安定した転送量を削減できるという特徴を持つ.
  \item 提案手法実現のために,
  MongoDB 上のデータ整合性を保証しつつ B+Tree のリーフページを選択的に解放するキャッシュクリア機構,
  MongoDB・QEMU 間のセマンティクスギャップを解消し解放されたページの転送を制御する連携機構,
  移送先ホストで転送をスキップしたページを透過的にストレージから再取得する復元機構
  を設計した.
  \item 提案手法を QEMU 10.2.0,MongoDB 8.2.2 に実装し,
  最大で総移送時間を 56\%,メモリ転送量を 67\% 削減することに成功した.
\end{itemize}

本論文では,第 2 章で本研究の背景となる仮想化技術や仮想マシンライブ移送,
ドキュメントデータベースについて述べ,それに関する問題についても述べる.
第 3 章では関連研究について紹介し,本研究との違いを述べる.
第 4 章で本研究の提案について述べ,そのアプローチや技術的課題を説明する.
第 5 章で提案手法の設計について述べ,第 6 章でその実装について述べる.
第 7 章では提案手法を適用して実験を行い,その結果について述べる.
第 8 章でまとめと今後の課題について述べる.

\chapter{背景}

\section{クラウド仮想化}

\section{仮想化環境}

\section{Unikernel}

\chapter{関連研究}
%A Survey on Virtual Machine Migration:Challenges, Techniques and Open Issues
%4章a
\section{圧縮転送}

\subsection{Delta Compression}

\subsection{Adaptive Memory Compression}


\section{vCPU throttling}

\subsection{Optimizing live migration by CPU scheduling}

\subsection{XvMotion}

\subsection{AdaMig}


\section{アプリケーションのセマンティクスを利用した転送削減}

\subsection{JAVMM}


\section{移送の並列化}

\subsection{PMigrate}


\section{まとめ}

\chapter{提案}

\section{概要}
2.4 節で述べたように,MongoDB のような DDB を実行している VM のライブ移送では,
メモリサイズやダーティレートが移送の性能に大きな影響を与える.
特に,Pre-copy 手法においてダーティレートが転送速度を上回る場合,
反復転送が収束せずにダウンタイムの増大やサービス品質の低下を招く点が課題である.

そこで,本研究では DDB が稼働している仮想マシンを対象としたライブ移送を高速化する手法を提案する.
提案手法では,VM のメモリサイズやダーティレートがライブ移送の性能に
影響を与える主要因の一つであることに着目し,
DDB が保持するキャッシュを解放してそれらを転送対象から除外し,
従来の移送手法では転送されていた DDB 上のメモリ領域を削減することで,移送時間の短縮を図る.
本研究では共有ストレージ環境を前提としており,移送先 VM では転送をスキップした DDB 上の
キャッシュを共有ストレージから読み込むことで再構築を行う.
これにより DDB のメモリ使用量が大きく,ダーティレートが高い状況下でも
転送すべきデータ量を少量に保つことができ,
総移送時間とダウンタイム双方の短縮を実現する.
その結果 DDB が稼働している VM においても短時間での移送を実現し,
ライブ移送による負荷分散や電力削減といった効果の享受を可能にする.

\section{アプローチ}
提案手法の概略図を図\ref{fig:Post_copy}に示す.
本研究では信頼性が高い Pre-copy 手法を基に,VM 内の DDB アプリケーションと
 VMM が連携して移送の効率化を図る.具体的な手順としてはまず,
移送を開始する直前に DDB が自身のメモリ管理機構を用いてキャッシュの解放を行う.
このとき,解放したページのアドレス情報を VMM に通知する.
移送元ホストの VMM は,受け取ったページ情報をもとにページの転送可否を判定するデータ構造を作成する.
実際のライブ移送中において VMM はこの情報を参照し,DDB 上のキャッシュ領域の転送をスキップする.
そして,移送先ホストにおける VM の実行状態は,移送元ホストからのネットワーク転送と
共有ストレージからのストレージアクセスを組み合わせて復元される.

\begin{figure}[H]
  \centering
  \includegraphics[width=0.8\linewidth]{figure/Proposal.png}
  \caption{提案手法の概略図}
  \label{fig:Proposal}
\end{figure}

\section{Design Challenges}
提案手法を実現する上で解決すべき,設計上の課題を述べる.

\subsection*{DDB の内部状態を考慮した安全なキャッシュ解放}%(→ 先にストレージにフラッシュする)
%\subsubsection*{どのようにして DDB 上の安全にデータキャッシュを解放するか}%(→ 先にストレージにフラッシュする)
%ここは, 1 パラグラフ 1 トピックが守られているか分からない.問題提起->課題なのか問題点->技術的課題とするか
DDB のキャッシュ領域には,ディスクと同期済みであるクリーンページ,
ページ内容が更新されたダーティページ,インデックスなどのメタデータが混在している.
ゲスト OS や VMM などの DDB 外部からはこれらのページのセマンティクスを理解していないため,
無差別に DDB 上のキャッシュを解放すると DDB 内部の整合性を破壊し,
プロセスのクラッシュやデータ損失を引き起こすリスクがある.
したがって,DDB 内部から自身のキャッシュ状態を判断し,
安全に解放する必要がある.これを実現するにあたり 2 つの課題が存在する.

 1 つ目は,更新データの整合性保証である.
ダーティページを単にメモリから解放した場合,変更内容はディスク上に反映されないため,
移送先ホストでそのデータを読み込んだ際に,直近の更新内容が消失してしまう.
そのため,解放するメモリ上の更新内容を共有ストレージに反映させたうえで,解放を行う必要がある.
しかし,ダーティページを個別に解放しようとすると,その都度ディスクへの書き戻しが発生し,
多大な I/O オーバーヘッドによって移送時間の短縮効果が相殺されてしまう.
したがって,効率的に更新データの整合性を保証しつつ,安全にキャッシュを解放することが求められる.

 2 つ目は,データ構造の依存関係の維持である.
例えば B+Tree などの DDB 内のキャッシュ管理構造において,内部ノードを誤って解放すると
リーフノードへのアクセスが不可能になり,データベースが機能しなくなる.
そのため,依存関係に影響を及ぼさないページのみを選定して解放しなければならない.


\subsection*{どのようにしてデータキャッシュのページ転送をスキップするか}%(->DB-OS-Hypervisor を連動させてアドレスを変換する,メモリの反復転送の際に転送の可否を判定)
%\subsubsection*{どのようにしてデータキャッシュのページ転送をスキップするか}%(->DB-OS-Hypervisor を連動させてアドレスを変換する,メモリの反復転送の際に転送の可否を判定)

本提案機構は,DDB と VMM 間のメモリ管理におけるセマンティックギャップを解消し,
VMM のメモリ転送機構に低遅延で統合することが求められる.

これを実現するにあたり,まず DDB と VMM 間アドレスの不一致を解消する必要がある.
VM 上で動作する DDB はゲスト仮想アドレス (GVA) でメモリを管理する一方,
VMM はゲスト物理アドレス (GPA) やホスト仮想アドレス (HVA),内部のメモリ管理構造を用いてメモリの管理を行っている.
DDB が扱う GVA はそのプロセス内でのみ有効な値であり,VMM がこれを直接解釈することはできない.
もし,未変換のアドレスを VMM でそのまま使用してしまうと,
カーネル領域などの誤ったメモリ領域をスキップ対象と誤認し,データが破損する恐れがある.
そのため,DDB と VMM 間で連携するには,アドレス変換を行ってセマンティクスギャップを埋めることが求められる.

また,VMM のメモリ転送機構に対して,オーバヘッドを最小限に抑えつつ転送スキップ処理を統合することが求められる.
キャッシュを解放したページアドレスのリストを DDB から適切に受信しても,
ライブ移送中のメモリ転送の際に,DDB から通知された膨大な転送スキップリストで毎回判定すると,
転送ループごとの判定コストが増加するため,転送スループットの低下や総移送時間が長期化を招く.
そのため,通知されたリストを VMM でビットマップ等の瞬時に参照可能な形式に変換し,
それを用いてスキップ判定を行うことで,移送を効率化する必要がある.

%なぜ解放するだけでなく,qemuにアドレスを通知して転送をスキップする必要があるのか?
%qemuはmongoのキャッシュを理解していないため,mongoDBで解放してもqemu側ではまだ必要だと思っている

%QEMUはゲストの物理アドレスしか理解できないため、MongoDBはキャッシュをクリアする際、対象となるページの仮想アドレス(GVA)を物理アドレス(GPA)に変換が必須となります。

\subsection*{どのようにしてストレージから復元するか}%(-> DDB のセマンティクスを利用してファイルから該当データを取得する)
%\subsubsection*{どのようにしてストレージから復元するか}%(-> DDB のセマンティクスを利用してファイルから該当データを取得する)
 3 つ目の課題は,転送をスキップしたページが移送先ホストで適切にアクセス可能であり,
アプリケーションから見て透過的に復元されることを保証することである.
本手法では,DDB 上のキャッシュ部分のメモリページを転送しないため,
移送先ホストでは該当ページが転送を正常にスキップされたことを認識する必要がある.
また,該当部分は移送先ホストの VM 上に存在しない.そのため,
移送先で DDB が該当データにアクセスした際,DDB のセマンティクスを利用して
自動的に共有ストレージから該当データを取得する必要がある.

\chapter{設計}

\section{DDB 上のキャッシュクリア}

\section{DDB と ホスト OS 間の連携とページ転送の判定}

\section{キャッシュの復元}
\chapter{実装}
本章では提案手法の実装について述べる.
本研究では,VMM として QEMU 10.2.0 上に,DDB として MongoDB 8.2.2 上に実装を行った.

\section{mongoDB 上のキャッシュクリア機構の実装}
提案機構では,キャッシュ解放フェーズに入る前の準備として,
WiredTiger の標準 API である WT\_SESSION::checkpoint を呼び出してチェックポイント処理を実行する.
これにより,一括でキャッシュ上の全ての変更をディスクに反映し,全ページをクリーンな状態とする.

キャッシュ解放処理の開始時には自作の制御フラグを有効化することで,
後述する既存のキャッシュ退避関数において,通常の LRU アルゴリズムとは異なる
本提案手法のアルゴリズムへ分岐させる.
解放プロセスでは,退避候補ページの収集とページ退避を行う関数を反復的に呼び出すことにより,
キャッシュ内に存在する可能な限り多くのリーフページの解放し,それらに対応する GPA の収集を行う.

\subsection{退避候補ページの収集}
退避候補ページは主に \_\_evict\_lru\_walk(),そこから呼び出される\_\_evict\_walk(),
さらにここから呼び出される \_\_evict\_walk\_tree() を通じて収集される.
本提案機構ではこれら既存の関数に対して,
探索制限の撤廃とリーフページの限定収集を行うように実装した.

\_\_evict\_lru\_walk() は退避キュー補充処理の最上位の関数であり,
本来は \_\_evict\_walk() を呼び出してキューに候補ページを補充させた後,
アクセス頻度や重要度によって並び替えを行い,
キューに残ったページの上位半数程度を実際の退避候補として選定する役割を担う.
本提案機構では,この選定において候補ページの切り捨てを無効化し,
下位関数で収集されたページをそのまま退避対象とする.

\_\_evict\_walk() は,主にページの探索を行う B+Tree を選定する関数である.
本来 WiredTiger は効率性と探索最適化のために,
キャッシュ使用率の低い B+Tree や,最近アクセスされた B+Tree などに対して
探索をスキップする仕組みを持つが,本提案機構ではこれらを無効化する.
これにより,全ての B+Tree 内を探索対象とし,あらゆるリーフページを網羅的に収集する.

\_\_evict\_walk\_tree() は,選定された B+Tree 内を走査し,
退避候補となるページを収集する関数である.
本提案機構では,収集数や探索回数に基づくヒューリスティックな探索打ち切り判定を無効化し,
B+Tree 内の全てのページを探索するか,キューが満杯になるまで探索を継続する.
また,ページの収集時にページの種類を判別し,内部ページを収集対象から除外する.
これにより,リーフページだけを選別し,最大限退避キューに収集する.

\subsection{ページの解放}
退避キューに収集されたページは \_\_evict\_page() によって 1 ページずつ退避される.
本提案機構では,この関数内でページを退避する前にそのページに対応する GPA を特定し,
退避が成功した際にそれらを多段リスト構造に追加する処理を実装した.

アドレスを収集する対象となる領域は,ディスクイメージと更新リストのアドレス情報である.
ディスクイメージはページ管理構造体 WT\_PAGE のメンバ dsk によって参照される.
dsk はヘッダ情報とデータ本体を含む連続したメモリブロックの先頭アドレスを指しており,
本提案機構ではこのブロック全体を転送スキップ対象とする.
図\ref{fig:WT_UPDATE}に,MongoDB における更新リストの管理方法を示す.
MongoDB において,更新リストはページ内の各行に対応したポインタ配列 mod\_row\_update と,
そこから各行の変更履歴をリストとして保持する WT\_UPDATE 構造体によって管理されている.
そのため,ポインタ配列自体と各行の変更データを保持する WT\_UPDATE 構造体のリストを
転送スキップ対象とする.

\begin{figure}[H]
  \centering
  \includegraphics[width=0.7\linewidth]{figure/WT_UPDATE.png}
  \caption{更新リストの管理}
  \label{fig:WT_UPDATE}
\end{figure}

図\ref{fig:Page_Alignment}に,対象領域における転送スキップページの決定方法を示す.
転送スキップ候補となるメモリブロックに対しては,
対象データがページ全体を完全に占有しているページの先頭アドレスを登録する.
1つのページ内に対象データと他のデータが混在している場合,
そのページの転送をスキップすることで,データの不整合が生じるリスクがある.
そのため,対象領域の開始アドレスを次のページ境界まで切り上げ,
終了アドレスを前のページ境界まで切り捨てることでアライメントを行う.
そしてその間に含まれるページのみを抽出することで,
データの端部が含まれるページやサイズが1ページに満たないページを除外し,
安全に転送をスキップできるページのみを抽出する.

\begin{figure}[H]
  \centering
  \includegraphics[width=0.5\linewidth]{figure/Page_Alignment.png}
  \caption{対象領域における転送スキップページの決定}
  \label{fig:Page_Alignment}
\end{figure}

抽出したページの先頭アドレスから GPA へのアドレス変換は,自作関数 gva\_to\_gpa() で行う.
この関数を,コード\ref{code:gva_to_gpa}に示す.
Linux では,/proc/self/pagemap を通じてユーザ空間から仮想メモリと
物理メモリのマッピング情報を参照する.
そのため,gva\_to\_gpa() はこのページテーブルを用いてアドレス変換を行う.
また,関数内ではページテーブルから読み込むシステムコール (pread) のオーバヘッドを減らすために,
一度の呼び出しで 512 エントリ分をまとめて読み込み,
pagemap\_buffer にキャッシュとして保持している.
変換処理では,読み取った 64 ビットの pfn\_item を解析することで変換を行う.
まず最上位ビットを参照することで,ページが物理メモリ上に存在するかを確認し,
下位 55 ビットの物理ページ番号 (PFN) を抽出する.
そして,この PFN とページサイズの積を求め,
この値に元の GVA のページ内オフセットを加算することで GPA を求める.

\begin{lstlisting}[caption=gva\_to\_gpa(), label=code:gva_to_gpa, language=C]
uint64_t pagemap_buffer[512];
uintptr_t buffer_base_vpn = (uintptr_t)-1;

uintptr_t gva_to_gpa(void *vaddr) {
    uint64_t vpn = (uintptr_t)vaddr / 4096;
    uint64_t pfn_item = 0;

    if (buffer_base_vpn != -1 && vpn >= buffer_base_vpn && 
        vpn < buffer_base_vpn + PAGEMAP_CACHE_COUNT) {
        pfn_item = pagemap_buffer[vpn - buffer_base_vpn];
    } else {
        buffer_base_vpn = (vpn / 512) * 512;
        uint64_t offset = buffer_base_vpn * 8;

        if (pread(pagemap_fd, pagemap_buffer, sizeof(pagemap_buffer), offset) < 8) {
            buffer_base_vpn = -1; return 0;
        }
        pfn_item = pagemap_buffer[vpn - buffer_base_vpn];
    }

    if ((pfn_item & (1ULL << 63)) == 0) return 0;    
    uint64_t pfn = pfn_item & ((1ULL << 55) - 1);
    return (pfn * 4096) + ((uintptr_t)vaddr % 4096);
}
\end{lstlisting}

転送スキップ対象となるメモリ領域の GPA を特定した後,
既存の下位関数である \_\_wt\_evict() を呼び出してページの退避を実行する.
そしてページの解放に成功した場合に限り,特定した GPA を多段リスト構造の最下層に登録することで,
転送が不要となった安全なページのみをスキップ対象として管理する.

\section{mongoDB と QEMU 間の連携機構の実装}
本節では,6.1 節で MongoDB 側において収集された転送スキップ対象ページの GPA リストを,
QEMU 側で受け取りビットマップへ反映させるまでの実装について述べる.
QEMU 側での処理は大きく分けて,MongoDB からの通知を受け取る受信処理と,
受け取った GPA を解析して QEMU が管理する形式のビットマップを構築する
アドレス解析処理の 2 段階で構成される.

\subsection{I/O ポートによる通知とボトムハーフ機構}
6.1 節で収集した転送をスキップする GPA リストの通知は,
I/O ポートを用いたハイパーコールにより実装する.
MongoDB は多段リスト構造のルート GPA を I/O ポート経由で QEMU に渡す.
このとき,64 ビットのルート GPA を 32 ビットに分割してデータ用ポートへ書き込み,
最後にトリガー用ポートへの書き込みを行う計 3 回の outl 命令によって
QEMU 側に制御を移し,MongoDB 側の通知処理を完了する.

QEMU 側の受信処理では,QEMU のボトムハーフ機構により非同期処理を実現する.
MongoDB からの通知を受け取る I/O ポートハンドラ内では,
受信したルート GPA を QEMU 側の変数に格納し,
qemu\_bh\_schedule() を呼び出してボトムハーフの実行を予約してゲスト OS に制御を戻す.
これにより,vCPU スレッド内での処理を最小限に抑え,
メモリ参照やアドレス変換を伴う,高コストなアドレス解析処理を
メインスレッド側で非同期的に実行することで,
ゲスト OS の実行性能への影響を最小限にする.

\subsection{アドレス解析機構}
アドレス解析処理ではまず,保持していたルート GPA を起点として多段リスト構造を順次読み込み,
最下層の転送スキップ対象ページの GPA を特定する.
この処理を,自作関数 resolve\_skip\_address() によって実装する.
この関数をコード\ref{code:resolve_skip_address}に示す.

GPA と QEMU プロセス内の HVA はアドレス空間が異なる.
そのため関数内では,cpu\_physical\_memory\_read() を用いて,
HVA へのアドレス変換を介してそのデータを読み込み,ローカル変数に複製する.
この処理を最上位層,中間層と順に行い,最下層のスキップ対象ページの GPA リストを特定し,
後述する update\_skip\_bitmap() にそのアドレスを渡す.
このように,GPA が格納されている各階層のリストを QEMU 内のバッファに格納して,
MongoDB で管理していたリスト構造を QEMU 側で適切に解釈することで,
MongoDB と QEMU 間のセマンティクスギャップを解消する.

\begin{lstlisting}[caption=resolve\_skip\_address(), label=code:resolve_skip_address, language=C]
typedef struct {
    uint64_t entry_count;
    uint64_t next_gpas[510];
} Directory_Node;

void resolve_skip_address(hwaddr root_gpa) {
    Directory_Node root;
    cpu_physical_memory_read(root_gpa, &root, sizeof(root));

    for (int i = 0; i < root.entry_count; i++) {
        hwaddr mid_gpa = root.next_gpas[i];
        if (mid_gpa == 0) continue;
        
        Directory_Node mid;
        cpu_physical_memory_read(mid_gpa, &mid, sizeof(mid));
        for (int j = 0; j < mid.entry_count; j++) {
            if (mid.next_gpas[j] != 0) update_skip_bitmap(mid.next_gpas[j]);
        }
    }
}
\end{lstlisting}

特定した転送スキップ対象ページのリストに含まれる各々の GPA に対して,
QEMU の移送制御で用いられる管理形式へと変換を行う.
この処理を行う自作関数 update\_skip\_bitmap() をコード\ref{code:update_skip_bitmap}に示す.
QEMU ではメモリを RAMBlock 構造体によってブロック単位で管理しており,
各ページは RAMBlock ごとのオフセットに基づくビットマップによって転送の可否が判定されている.
そのため,まず cpu\_physical\_memory\_map() を用いて転送スキップ対象の GPA を HVA に変換し,
得られた HVA を qemu\_ram\_block\_from\_host() に渡して
該当する RAMBlock 内オフセットを特定する.

そして,アドレス変換によって得られた RAMBlock 内オフセットをもとに
スキップ用のビットマップを構築する.
この際,特定した RAMBlock が メインメモリ領域 (pc.ram) であることを確認する.
本実装では,MongoDB 上のキャッシュはメインメモリ領域から割り当てられることを前提としており,
この判定により他のメモリ領域への誤った操作を防ぐ.
ビットマップ登録処理では,特定された RAMBlock 内オフセットをページサイズで割り,
ビットマップ上のインデックス (ページ番号) に変換し,
ビットマップ内の該当ビットをセットする.
最後に,cpu\_physical\_memory\_unmap() を呼び出してアドレスマッピングを解放する.
以上の一連の処理により,転送スキップ対象ページを確定させる.

\begin{lstlisting}[caption=update\_skip\_bitmap(), label=code:update_skip_bitmap, language=C]
typedef struct {
    uint64_t count;
    uint64_t total_pages;
    uint64_t gpa_list[510];
} SKIP_PAGE_GPA_LIST;

static void update_skip_bitmap(uint64_t batch_gpa) {
    SKIP_PAGE_GPA_LIST gpa_list;
    cpu_physical_memory_read(batch_gpa, &gpa_list, sizeof(gpa_list));

    for (uint64_t i = 0; i < gpa_list.count; i++) {
        hwaddr target_gpa = gpa_list.gpa_list[i];
        hwaddr len = TARGET_PAGE_SIZE;
        void *hva = cpu_physical_memory_map(target_gpa, &len, false);
        
        ram_addr_t offset_in_block;
        RAMBlock *block = qemu_ram_block_from_host(hva, false, &offset_in_block);
            
        if (block && strcmp(block->idstr, "pc.ram") == 0) {
            uint64_t page_idx = offset_in_block >> TARGET_PAGE_BITS;
            test_and_set_bit(page_idx, global_skip_bitmap);
        }
        cpu_physical_memory_unmap(hva, len, false, len);
    }
}
\end{lstlisting}

\section{転送をスキップするページの判定と受信処理}
Pre-copy 手法の反復転送処理において,ram\_save\_host\_page() でダーティページの転送を行っているため,
ここに MongoDB から通知されたページの転送を制御する実装を行う.
具体的には,ダーティビットマップにより転送が必要だと判定されたページに対し,
ダーティビットをクリアしたうえで,さらにスキップ用のビットマップによる判定を加える.
本実装ではメインメモリ領域 (pc.ram) を対象とするため,
対象ページが pc.ram ブロックに属しており,スキップ用のビットマップに登録されているという条件を満たした場合,
通常の 4KB の実データ転送を行わず,
新たに追加した RAM\_SAVE\_FLAG\_SKIPPED フラグを含む通常 8 バイトのヘッダ情報を転送する.
これにより,不要なページデータの転送に伴う帯域消費と CPU コストを削減し,
反復転送処理の高速化を実現する.

移送先ホスト側において RAM\_SAVE\_FLAG\_SKIPPED フラグを含むヘッダ情報を受けとった場合,
メモリへの書き込み処理を行わずに受信処理を終了する.
これにより,転送をスキップしたページは初期状態として維持されるため,
移送先ホストでの実行再開後,WiredTiger によって透過的にキャッシュ再構築が行われる.

\chapter{実験}
本章では,提案手法による VM ライブ移送パフォーマンスの実験について述べる.

\section{実験環境}
実験環境は移送を行うホストマシン 2 台,共有ストレージ用のマシン 1 台から構成される.
実験環境の概要図を図\ref{fig:exp_machine}に,各マシンの構成を表\ref{tab:machine_spec}に示す.
本実験は,同一ハードウェア構成を持つ 2 台の移送用マシンと,
VM のディスクイメージを共有するための NFS サーバマシンで構成される.
移送用マシンの CPU は Intel Xeon E-2124 (3.30 GHz, 4 コア 4 スレッド) を,
メモリは 64GB を搭載している.
共有ストレージ用マシンの CPU はIntel Xeon Gold 5215 (2.50 GHz, 10 コア 20 スレッド) を,
メモリは 128GB を搭載している.
全てのマシンは 1 Gbps のイーサネットで接続されており,
OS として Ubuntu 24.04 (Linux 6.8.0) が稼働している.
VMM としては QEMU/KVM 10.2.0 をベースに提案機構を実装したものを使用する.
ライブ移送の対象となる VM には 2 vCPU を割り当て,
ゲスト OS はホストマシン同様,Ubuntu 24.04 (Linux 6.8.0) が稼働している.
また,VM のメモリサイズは本実験におけるパラメータとして変化させる.
そして,VM 上では DDB として提案機構を組み込んだ MongoDB 8.2.2 を動作させる.

\begin{figure}[H]
  \centering
  \includegraphics[width=0.8\linewidth]{figure/exp_machine.png}
  \caption{実験環境の概要図}
  \label{fig:exp_machine}
\end{figure}

\begin{table}[H] % [H]でその場所に固定、[htbp]なら自動配置
  \centering
  \caption{実験に使用した各マシンの構成} % 表のタイトルは「上」に書く
  \label{tab:machine_spec} % 参照用のラベル
  \begin{tabular}{ccc} \toprule % lは「左寄せ」cは「中央」の意味
    マシン & CPU & メモリ \\ \midrule\midrule
    ホストマシン     & Intel Xeon E-2124 CPU @ 3.30GHz (4C/4T) & 64GB \\
    ストレージマシン & Intel Xeon Gold 5215 CPU @ 2.50GHz (10C/20T) & 128GB \\
    仮想マシン      & 2vCPU & 可変 \\ \bottomrule
  \end{tabular}
\end{table}

\section{実験方法}
提案手法によるライブ移送を従来の Pre-copy 手法と比較し,性能を評価する.
実験の初期状態として,移送開始前に VM 上で動作する MongoDB の
WiredTiger キャッシュサイズを VM メモリ割り当て量の 95\% に設定する.
そして,キャッシュ内のダーティページ占有率が 95\% となるように書き込みと更新を行うことで,
VM 上で動作する MongoDB の大量データがメモリ上に保持されている状況を生成する.
この書き込み処理終了後のアイドル状態の VM に対して移送を実行し,
その際の総移送時間,メモリ転送量,転送ページ数,CPU 時間,ダウンタイムを計測する.
以上の移送実験において,両手法において VM に 8GB, 16GB, 32GB, 48GB, 60GB と,
割り当てるメモリサイズを変化させて実験を行う.
また,各条件での実験を計 5 回行い,それらの平均の実験結果として採用する.

\section{実験結果}
\subsection{総移送時間}
実験における総移送時間の結果を図\ref{fig:exp_migration_time}に示す.
実験結果より,全てのメモリサイズにおいて,
総移送時間は従来の Pre-copy 手法と比較して大幅に短縮され,約半分の時間で移送を完了した.
この総移送時間には,提案手法における MongoDB 上のキャッシュを解放する時間も含まれているため,
キャッシュ解放によるオーバヘッドを加味しても,
移送プロセス全体の大幅な高速化が実現できているといえる.
また,メモリサイズが大きくなるにつれて時間短縮の効果が顕著になり,
メモリサイズが 60GB の条件下では最大約 56\% の短縮を達成した.
このことから,特にメモリサイズが大きい環境において,提案手法の優位性が高まると考えられる.

\begin{figure}[H]
  \centering
  \includegraphics[width=0.8\linewidth]{figure/exp_migration_time.png}
  \caption{総移送時間}
  \label{fig:exp_migration_time}
\end{figure}


\subsection{メモリ転送量}
実験におけるメモリ転送量の結果を図\ref{fig:exp_transffer_data}に,
転送ページ数の結果を図\ref{fig:exp_transffer_page}に示す.
実験結果より,全てのメモリサイズにおいて提案手法はメモリ転送量,
転送ページ数ともに Pre-copy 手法と比較して大幅に削減できていることが確認された.
また,メモリ転送量と転送ページ数の削減率が概ね一致していることから,
提案手法による解放したキャッシュのページ転送スキップが,
転送の削減に寄与しているといえる.
そして,メモリサイズが大きくなるにつれて転送削減効果が増し,
メモリサイズが 60GB の条件下では最大約 67\% の削減を達成した.
このことから,MongoDB のキャッシュに割り当てられるメモリサイズが大きくなるほど,
提案手法の効果が発揮されると考えられる.

\begin{figure}[H]
  \centering
  \includegraphics[width=0.8\linewidth]{figure/exp_transffer_data.png}
  \caption{メモリ転送量}
  \label{fig:exp_transffer_data}
\end{figure}

\begin{figure}[H]
  \centering
  \includegraphics[width=0.8\linewidth]{figure/exp_transffer_page.png}
  \caption{転送ページ数}
  \label{fig:exp_transffer_page}
\end{figure}


\subsection{CPU 時間}
実験におけるCPU 時間の結果を図\ref{fig:exp_cpu_time}に示す.
実験結果によると,全てのメモリサイズにおいて提案手法は Pre-copy 手法と比較して,
CPU 時間を一貫して削減できていることが確認された.
具体的には,どのメモリサイズにおいても 概ね 15\% 前後の削減率を示しており,
メモリサイズが 16GB の条件においては,最大約 21\% の削減を達成した.
提案手法では,MongoDB 側でのキャッシュ解放と,QEMU 側での転送スキップビットマップ構築
という追加の CPU 処理が発生する.
しかし,これらのオーバヘッドを加味してもページ転送処理の削減効果が上回るため,
結果的にトータルの CPU 時間が削減されていると考えられる.
このことから,メモリサイズに関わらず安定して従来手法より CPU 負荷が低く抑えられており,
メモリサイズが増加しても提案手法の CPU に対する効率性が損なわれないと考えられる.

\begin{figure}[H]
  \centering
  \includegraphics[width=0.8\linewidth]{figure/exp_cpu_time.png}
  \caption{CPU 時間}
  \label{fig:exp_cpu_time}
\end{figure}


\subsection{ダウンタイム}
実験におけるダウンタイムの結果を図\ref{fig:exp_downtime}に示す.
実験結果によると,全てのメモリサイズにおいて提案手法は Pre-copy 手法と比較して,
ダウンタイムを一貫して削減できていることが確認された.
特に,メモリサイズが 48GB の条件においては,最大約 42\% の削減を達成した.
これは,解放されたキャッシュの転送スキップにより,
Stop-and-Copy フェーズで転送すべきページ数が抑制されたためだと考えられる.
このことから,提案手法はどのメモリサイズにおいてもダウンタイムを抑制し,
サービス停止による影響を最小限に抑えることができると考えられる.

\begin{figure}[H]
  \centering
  \includegraphics[width=0.8\linewidth]{figure/exp_downtime.png}
  \caption{ダウンタイム}
  \label{fig:exp_downtime}
\end{figure}


\chapter{おわりに}
\section{まとめ}
\section{今後の課題}

% ページ番号を小文字アルファベットに
\pagenumbering{alph}

% 謝辞
\backmatter %後付け(章番号をつけない)

\chapter{謝辞}
本研究を進めるにあたり,指導教官である山田浩史准教授には
お忙しい中多くのことをご教授いただきました.
ご多忙の中,研究や講義の合間を縫って毎週のミーティングを実施していただき,
研究方針や研究におけるアドバイスなどを頂けたことで,
自信をもって研究を円滑に進めることができました.
重ねて感謝申し上げます.
また,研究室の先輩方には,研究進め方や論文執筆に関してご助言をいただき,
研究活動において大きな力となりました.
同期の皆様とは,研究に関して互いに知見を共有することで,
充実した研究生活を送ることができました.
皆様に心よりお礼申し上げます.


% 参考文献
%\nocite{*}
% BibLaTeX の場合
\printbibliography[heading=bibintoc,title=参考文献]
% BibTeX の場合
%\bibliographystyle{junsrt}
%\bibliography{citation/reference}

% 付録
%%\clearpage
%\appendix
%\chapter*{対外発表}
%\addcontentsline{toc}{chapter}{付録}

\backmatter%
\appendix
\chapter*{対外発表}

\begin{description}
\item[【F-1】]\underline{金津 穂},田崎 創,宇夫 陽次郎,山田 浩史:

   ``クラウド環境を指向するライブラリ OS の分類のための起動にかかる時間の計測'',
		
  Internet Conference 2016,Oct.\ 2016.\
		
\item[【B-1】]\underline{Minoru KANATSU},Hiroshi YAMADA\@:

    ``Running Multi-Process Applications on Unikernel-based VMs'',
		
    \textit{\nth{26} ACM Symposium on Operating Systems Principles (SOSP '17)}, Oct.\ 2017.\

%	\item[(1)] 筆者:
%		タイトル,
%		学会名など(年).
\end{description}


\end{document}
